%%%%%%%%%%%%%%%%%%%%%%%%%%%%%%%%%%%%%%%%%%%%%%%%%%%%%%%%%%
%   Autoren:
%   Prof. Dr. Bernhard Drabant
%   Prof. Dr. Dennis Pfisterer
%   Prof. Dr. Julian Reichwald
%%%%%%%%%%%%%%%%%%%%%%%%%%%%%%%%%%%%%%%%%%%%%%%%%%%%%%%%%%

%%%%%%%%%%%%%%%%%%%%%%%%%%%%%%%%%%%%%%%%%%%%%%%%%%%%%%%%%%
%	ANLEITUNG: 
%   1. Ersetzen Sie firmenlogo.jpg im Verzeichnis img
%   2. Passen Sie alle Stellen im Dokument an, die mit 
%      @stud 
%      markiert sind
%%%%%%%%%%%%%%%%%%%%%%%%%%%%%%%%%%%%%%%%%%%%%%%%%%%%%%%%%%

%%%%%%%%%%%%%%%%%%%%%%%%%%%%%%%%%%%%%%%%%%%%%%%%%%%%%%%%%%
%	ACHTUNG: 
%   Für das Erstellen des Literaturverzeichnisses wird das 
%   modernere Paket biblatex in Kombination mit biber 
%   verwendet - nicht mehr das ältere Paket BibTex!
%
%   Bitte stellen Sie Ihre TeX-Umgebung entsprechend ein (z.B. TeXStudio): 
%   Einstellungen --> Erzeugen --> Standard Bibliographieprogramm: biber
%%%%%%%%%%%%%%%%%%%%%%%%%%%%%%%%%%%%%%%%%%%%%%%%%%%%%%%%%%

\documentclass[fontsize=12pt,BCOR=5mm,DIV=12,parskip=half,listof=totoc,
               paper=a4,toc=bibliography,pointlessnumbers]{scrreprt}
               
\usepackage[utf8]{inputenc}

%% LANGUAGE SETTINGS
%
% @stud: Sprache ggf. anpassen
%

%%%%%%%%%%%%%
%% ZITIERSTIL
%%%%%%%%%%%%%
%
% @stud: Zitierstil in package biblatex unten wählen
%
% NUMERIC Style - e. g. [12]
% style=numeric 
%
% IEEE Style - numeric kind of style 
% style=ieee 
%
% ALPHABETIC Style - e. g. [AB12]
% style=alphabetic 
%
% HARVARD Style 
% style=apa 
%
% CHICAGO Style 
% style=authoryear
%
% Position des Zitats:
%
% autocite=inline 
%
% (!!) FOOTNOTE POSITION NOT RECOMMENDED IN MINT DOMAIN:
% autocite=footnote
%
\usepackage[backend=biber, autocite=inline, style=numeric]{biblatex} 	
\usepackage{makeidx}                  % allows index generation
\usepackage{listings}	                %Format Listings properly
\usepackage{lipsum}                   % Blindtext
\usepackage{graphicx}                 % use various graphics formats
\usepackage[german]{varioref}         % nicer references \vref
\usepackage{caption}	                % better Captions
\usepackage{booktabs}                 % nicer Tabs
\usepackage[hidelinks=true]{hyperref} % keine roten Markierungen bei Links
\usepackage{fnpct}                    % Correct superscripts 
\usepackage{calc}                     % Used for extra space below footsepline, in particular
\usepackage{array}
\usepackage{acronym}
\usepackage{algorithm}
\usepackage{algpseudocode}
\usepackage{setspace}
\usepackage{tocloft}
\usepackage[T1]{fontenc}

% Definitionen und Commands
\newcommand{\indextype}{numeric}
\newcommand{\abs}{\par\vskip 0.2cm\goodbreak\noindent}
\newcommand{\nl}{\par\noindent}
\newcommand{\mcl}[1]{\mathcal{#1}}
\newcommand{\nowrite}[1]{}
\newcommand{\NN}{{\mathbb N}}
\newcommand{\imagedir}{img}
% \newcommand{\TitelDerArbeit}[1]{\def\DerTitelDerArbeit{#1}\hypersetup{pdftitle={#1}}}
% \newcommand{\AutorDerArbeit}[1]{\def\DerAutorDerArbeit{#1}\hypersetup{pdfauthor={#1}}}
% \newcommand{\Firma}[1]{\def\DerNameDerFirma{#1}}
% \newcommand{\Kurs}[1]{\def\DieKursbezeichnung{#1}}
% \newcommand{\Abteilung}[1]{\def\DerNameDerAbteilung{#1}}
% \newcommand{\Studiengangsleiter}[1]{\def\DerStudiengangsleiter{#1}}
% \newcommand{\WissBetreuer}[1]{\def\DerWissBetreuer{#1}}
% \newcommand{\FirmenBetreuer}[1]{\def\DerFirmenBetreuer{#1}}
% \newcommand{\Bearbeitungszeitraum}[1]{\def\DerBearbeitungszeitraum{#1}}
% \newcommand{\Abgabedatum}[1]{\def\DasAbgabedatum{#1}}
% \newcommand{\Matrikelnummer}[1]{\def\DieMatrikelnummer{#1}}
% \newcommand{\Studienrichtung}[1]{\def\DieStudienrichtung{#1}}
% \newcommand{\ArtDerArbeit}[1]{\def\DieArtDerArbeit{#1}}
% \newcommand{\Literaturverzeichnis}{Literaturverzeichnis}

% Page Layout
\oddsidemargin=0mm
\evensidemargin=0mm
\textwidth=159mm
\topmargin=-18mm
\headsep=10mm
\textheight=251mm
\footheight=15mm

\makeindex

%%%%%%%%%%%%%%%%%%%%%%%%%%%%%%%%%%%
% LITERATURVERZEICHNIS
% @stud: Literaturverzeichnis in Datei bibliography.bib anpassen. 
%
% [Alternative zu Verwendung von \initializeBibliography: Citavi ... (dazu eigenes LaTex Coding verwenden)]
%
\addbibresource{bibliography.bib}
\DefineBibliographyStrings{ngerman}{andothers = {{et\,al\adddot}},}

% Elementare Konfigurationen und Definitionen werden geladen 
% @stud: gegebenenfalls anpassen
%
\input{config}

% @stud
%
% PERSÖNLICHE ANGABEN (BITTE VOLLSTÄNDIG EINGEBEN zwischen den Klammern: {...})
%
\ArtDerArbeit{Seminar} % "Bachelor" oder "Projekt" wählen
\TitelDerArbeit{Emissionszertifikate als umweltpolitisches Instrument gegen den Klimawandel:
Theorie und Praxis}
\AutorDerArbeit{Iven Stahl}
\Abteilung{<Ihre Abrteilung>}
\Firma{<Ihre Firma>}
\Kurs{WWI23SEB}
\Studienrichtung{Software Engineering}
\Matrikelnummer{9820647}
\Studiengangsleiter{VWL Wirtschaftspolitik}
\WissBetreuer{Prof. Dr. Hubert}
\FirmenBetreuer{<Ihr(e) Firmenbetreuer(in)>}
\Bearbeitungszeitraum{14.05.2025 -- 26.07.2025}
\Abgabedatum{dd.mm.yyyy}

\begin{document}

\setTitlepage

%%%%%%%%%%%%%%%%%%%%%%%%%%%%%%%%%%%
% EHRENWÖRTLICHE ERKLÄRUNG
%
% @stud: ewerkl.tex bearbeiten
%
\input{ewerkl} 
\cleardoublepage  
%%%%%%%%%%%%%%%%%%%%%%%%%%%%%%%%%%%

%%%%%%%%%%%%%%%%%%%%%%%%%%%%%%%%%%%
% SPERRVERMERK
%
% @stud: nondisclosurenotice.tex bearbeiten
%
%\input{nondisclosurenotice} 
%\cleardoublepage
%%%%%%%%%%%%%%%%%%%%%%%%%%%%%%%%%%%

%%%%%%%%%%%%%%%%%%%%%%%%%%%%%%%%%%%
%	KURZFASSUNG
%
% @stud: acknowledge.tex bearbeiten
%
% \input{acknowledge}
% \cleardoublepage 
%%%%%%%%%%%%%%%%%%%%%%%%%%%%%%%%%%%

%%%%%%%%%%%%%%%%%%%%%%%%%%%%%%%%%%%
% VERZEICHNISSE und ABSTRACT
%
% @stud: ggf. nicht benötigte Verzeichnisse auskommentieren/löschen
%
\tableofcontents
\cleardoublepage

% Abbildungsverzeichnis
% \phantomsection
% \addcontentsline{toc}{chapter}{\listfigurename}
% \listoffigures
% \cleardoublepage

%	Tabellenverzeichnis
% \phantomsection
% \addcontentsline{toc}{chapter}{\listtablename}
% \listoftables
% \cleardoublepage

%	Listingsverzeichnis / Quelltextverzeichnis
% \lstlistoflistings
% \cleardoublepage

% Algorithmenverzeichnis
% \listofalgorithms
% \cleardoublepage

% Abkürzungsverzeichnis
% @stud: acronyms.tex bearbeiten
% \input{acronyms} 
% \cleardoublepage

\onehalfspacing

%	Kurzfassung / Abstract
% @stud: abstract.tex bearbeiten
\input{abstract} 
\cleardoublepage

\initializeText

%%%%%%%%%%%%%%%%%%%%%%%%%%%%%%%%%%%%%%%%%%%%%%%%%%%%%%%%%%%%%%%%%%%%%%%%%%%%%%%%%%%%%%%%%%
% KAPITEL UND ANHÄNGE
%
% @stud:
%   - nicht benötigte: auskommentieren/löschen
%   - neue: bei Bedarf hinzufügen mittels input-Kommando an entsprechender Stelle einfügen
%%%%%%%%%%%%%%%%%%%%%%%%%%%%%%%%%%%%%%%%%%%%%%%%%%%%%%%%%%%%%%%%%%%%%%%%%%%%%%%%%%%%%%%%%%

%%%%%%%%%%%%%%%%%%%%%%%%%%%%%%%%%%%
% KAPITEL
%
% @stud: einzelne Kapitel bearbeiten und eigene Kapitel hier einfügen
%
% Einleitung


% mehrere Grundlagen- und Forschungs-Kapitel
% !TEX root =  master.tex
\chapter{Beispiel-Kapitel: Gebrauchsanleitung \LaTeX}

\nocite{*}


% !TEX root =  master.tex
\chapter{Beispiel-Kapitel: Noch ein Kapitel}

test
% !TEX root =  master.tex
\chapter{Ökonomische Betrachtung des Emissionshandels}

\section{Effizienz und Preisbildung auf Zertifikatemärkten}

\section{Innovations- und Lenkungswirkungen des Zertifikatehandels}

\section{Problemfeld Carbon Leakage: Ökonomische Ursachen und Auswirkungen}






\nocite{*}
% !TEX root =  master.tex
\chapter{Praxisbeispiel: EU-Emissionshandelssystem (EU-ETS)}

\section{Funktionsweise und aktuelle Herausforderungen}

\section{Ökonomische Bewertung der bisherigen Umsetzung (Kosten, Nutzen und Carbon Leakage-Risiken)}







\nocite{*}
% !TEX root =  master.tex
\chapter{Kritische Würdigung und Reformvorschläge}

\section{Ökonomische Stärken und Schwächen des Emissionshandels}

\section{Ansätze zur Verringerung von Carbon Leakage}

\section{Vorschläge zur Steigerung der Effizienz und Wirksamkeit}






\nocite{*}
% !TEX root =  master.tex
\chapter{Fazit und Ausblick}



\nocite{*}

% Fazit und Ausblick
%\input{conclusion}
%%%%%%%%%%%%%%%%%%%%%%%%%%%%%%%%%%%

%%%%%%%%%%%%%%%%%%%%%%%%%%%%%%%%%%%
% ANHÄNGE
%
% @stud: einzelne Anhänge bearbeiten und eigene Anhänge hier einfügen 
%        die nachfolgenden Zeilen deaktivieren, wenn keine Anhänge verwendet werden
%
%\initializeAppendix
%\input{appendix1}
%\input{appendix2}
%%%%%%%%%%%%%%%%%%%%%%%%%%%%%%%%%%%

%\singlespacing

%\ihead{}
%\printbibliography[title=\Literaturverzeichnis] 
\printbibliography 
\cleardoublepage

%\initializeBibliography
%%%%%%%%%%%%%%%%%%%%%%%%%%%%%%%%%%%

%%%%%%%%%%%%%%%%%%%%%%%%%%%%%%%%%%%
% INDEX
% @stud: ggf. Index auskommentieren, wenn nicht benötigt
%
%\addcontentsline{toc}{chapter}{Index}
%\printindex

\end{document}
