% !TEX root =  master.tex
\chapter{Einleitung}

\section{Problemstellung und Aktualität}

Die Menschheit steht vor einer der größten Herausforderungen der Gegenwart: dem fortschreitenden Klimawandel.
Eine der Hauptursachen hinter dieser Entwicklung sind zunehmende Emissionen von Treibhausgasen, die in erster Linie durch wirtschaftliche Prozesse verursacht werden.
Besonders die energetische Nutzung fossiler Rohstoffe trägt signifikant zur Erwärmung des globalen Klimas bei und konfrontiert die Menschheit mit tiefgreifenden Herausforderungen.
Die Folgen eines ungebremsten Klimawandels sind gravierend und betreffen sowohl ökonomische als auch ökologische und gesellschaftliche Bereiche.
Ein zentrales Problem besteht darin, dass diese Emissionen negative Auswirkungen auf Dritte haben, ohne dass diese Kosten im Marktpreis berücksichtigt werden – es handelt sich um sogenannte externe Effekte. Dadurch versagt der Marktmechanismus, wenn es darum geht, umweltschädliches Verhalten angemessen einzudämmen.
Vor diesem Hintergrund ist es umso wichtiger, dass die Umwelt- und Wirtschaftspolitik geeignete Maßnahmen entwickelt, um Emissionen kosteneffizient zu senken. Eine zentrale Rolle spielen dabei marktwirtschaftliche Instrumente wie der Emissionshandel, der in der klimapolitischen Debatte zunehmend an Bedeutung gewinnt.

\section{Ziel und Aufbau der Arbeit}

Ziel dieser Arbeit ist es, das umweltökonomische Instrument des Emissionshandels aus volkswirtschaftlicher Sicht zu analysieren und sowohl dessen Funktionsweise als auch dessen praktische Anwendbarkeit kritisch zu hinterfragen.
Dafür werden zuerst kurz die Ökonomischen Ursachen des Klimawandels und die Prinzipien des Emissionshandels erläutert und anschließend mit anderen marktbasierten Instrumenten verglichen. 
In Kapitel 3 erfolgt darauf aufbauend eine vertiefte ökonomische Analyse, welche neben der Effizienz des Emissionshandels auch dessen Innovations- und Lenkungswirkungen sowie spezifische ökonomische Risiken des Carbon Leakage behandelt.
Anhand des EU-Emissionshandelssystems (EU-ETS) werden diese theoretischen Aspekte in Kapitel 4 praxisnah untersucht und kritisch bewertet.
Kapitel 5 bündelt die Erkenntnisse in einer Gesamtbewertung des Instruments, wobei neben den identifizierten Stärken auch bestehende Schwachstellen aufgezeigt werden wobei Reformmöglichkeiten und insbesondere Lösungsansätze zur Begrenzung von Carbon Leakage diskutiert werden.
Den Abschluss bildet Kapitel 6 mit einer kompakten Zusammenfassung der wichtigsten Erkentnisse sowie einem kurzen Ausblick auf künftige Entwicklungen im Bereich der Emissionspolitik.


\nocite{*}

